\documentclass{article}
\usepackage[T2A]{fontenc}
\usepackage{epigraph}
\usepackage[english, russian]{babel} % языковой пакет
\usepackage{amsmath,amsfonts,amssymb} %математика
\usepackage{mathtools}
\usepackage[oglav,spisok,boldsect,eqwhole,figwhole,hyperref,hyperprint,remarks,greekit]{../../style/fn2kursstyle}
\usepackage[utf8]{inputenc}
\usepackage[]{tkz-euclide}
\usepackage{algpseudocode}
\usepackage{pgfplots}
\usepackage{tikz-3dplot}
\usepackage[oglav,spisok,boldsect,eqwhole,figwhole,hyperref,hyperprint,remarks,greekit]{./style/fn2kursstyle}
\usepackage{multirow}
\usepackage{supertabular}
\usepackage{multicol}
\usepackage{tikz}
\usepackage{pgfplots}
\usepackage{float}
\usepackage{graphicx}
\pgfplotsset{compat=1.9}
\usepackage[svgnames]{pstricks}
\usepackage{pst-solides3d} 
\usepackage{multirow}
\usepackage{hhline}
\usepackage{slashbox}
\usepackage{pdflscape}
\usepackage{array} 
\graphicspath{{../../style/}{../}}

  


\newcommand{\cond}{\mathop{\mathrm{cond}}\nolimits}
\newcommand{\rank}{\mathop{\mathrm{rank}}\nolimits}
% Переопределение команды \vec, чтобы векторы печатались полужирным курсивом
\renewcommand{\vec}[1]{\text{\mathversion{bold}${#1}$}}%{\bi{#1}}
\newcommand\thh[1]{\text{\mathversion{bold}${#1}$}}
%Переопределение команды нумерации перечней: точки заменяются на скобки
\renewcommand{\labelenumi}{\theenumi)}
\newtheorem{theorem}{Теорема}
\newtheorem{define}{Определение}
\tdplotsetmaincoords{60}{115}
\pgfplotsset{compat=newest}

\title{Численное решение краевых задач для одномерного уравнения теплопроводности}
\author{Н.\,О.~Акиньшин}
\group{ФН2-61Б}
\date{2025}
\supervisor{А.\,С.~Джагарян}



\begin{document}
    \maketitle
    \newpage
    \tableofcontents
    \newpage

    \section{Контрольные вопросы}
	\begin{enumerate}
		\item Предложите разностные схемы, отличные от схемы «крест»,
		для численного решения задачи (3.1)–(3.4).
		\newline 
		{\bfseries Ответ. } 
		Рассмотрим задачу 
		\begin{equation*}
			\begin{cases}
				u_{tt} = a^2 u _{xx} \\
				u(x, 0) = f(x) \\
				u_t(x, 0) =  g(x) \\ 
				u(0, t) = \varphi(t) \\ 
				u(l, t) = \psi(t)
			\end{cases}
		\end{equation*}
		1. Схема Ричардсона (Крест)
		\[
		y_{t\overline{t}} = a^2 y_ {x\overline{x}}
		\]
		\[
		\frac{y_{i}^{j+1}-2y_i^{j}+y_i^{j-1}}{\tau^2} = a^2\frac{y_{i-1}^j -2y_i^j +y_{i+1}^j}{h^2}
		\]
		2. Схема Дюфорта - Франкел (ромб)
		В схеме Ричардсона нужно заменить $y_i^j = \frac{y_{i-1}^j + y_{i+1}^j}{2} = \frac{y_{i}^{j-1} + y_{i}^{j+1}}{2}$
		Получаем следующие схемы
		
		
		2.1) 
		\[
		\frac{y_{i}^{j+1}-2(\frac{y_{i-1}^j + y_{i+1}^j}{2}))+y_i^{j-1}}{\tau^2} = a^2\frac{y_{i-1}^j -2(\frac{y_{i}^{j-1} + y_{i}^{j+1}}{2})) +y_{i+1}^j}{h^2}
		\]
		2.2) 
		\[
		\frac{y_{i}^{j+1}-(2(\frac{y_{i-1}^j + y_{i+1}^j}{2})))+y_i^{j-1}}{\tau^2} = a^2\frac{y_{i-1}^j -2(\frac{y_{i}^{j-1} + y_{i}^{j+1}}{2})) +y_{i+1}^j}{h^2}
		\]
		
		
		3. Схема T(неявный крест)
		
		\[
		\frac{y_{i}^{j+1}-2y_i^{j}+y_i^{j-1}}{\tau^2} = a^2\frac{y_{i-1}^{j+1} -2y_i^{j+1} +y_{i+1}^{j+1}}{h^2}
		\]
		
		
		4.Схема обратная Т (абсолютно явный)
		\[
		\frac{y_{i}^{j+1}-2y_i^{j}+y_i^{j-1}}{\tau^2} = a^2\frac{y_{i-1}^{j-1} -2y_i^{j-1} +y_{i+1}^{j-1}}{h^2}
		\]
		
		
		5. Схема ступенька вверх
		\[
		y_{t\overline{t}} = a^2 \frac{\hat{y}_x - \check{y}_{\overline{x}}}{h}
		\]
		\[
		\frac{y_{i}^{j+1}-2y_i^{j}+y_i^{j-1}}{\tau^2} = \frac{a}{h^2} ((y_{i+1}^{j+1}-y_{i}^{j+1}) -(y_i^{j-1}- y_{i-1}^{j-1}))
		\]
		
		6. Схема ступенька вниз
		\[
		y_{t\overline{t}} = a^2 \frac{\check{y}_x - \hat{y}_{\overline{x}}}{h}
		\]
		\[
		\frac{y_{i}^{j+1}-2y_i^{j}+y_i^{j-1}}{\tau^2} = \frac{a}{h^2} ((y_{i+1}^{j-1} -y_{i}^{j-1} )-(y_i^{j+1} - y_{i-1}^{j+1}))
		\]
		
		
		7.Схема Почти ступенька вверх
		
		\[
		y_{t\overline{t}} = a^2 \frac{\hat{y}_x - {y}_{\overline{x}}}{h}
		\]
		\[
		\frac{y_{i}^{j+1}-2y_i^{j}+y_i^{j-1}}{\tau^2} = \frac{a}{h^2} ((y_{i+1}^{j+1}-y_{i}^{j+1}) -(y_i^{j}- y_{i-1}^{j}))
		\]
		
		8.Схема Почти ступенька вниз
		\[
		y_{t\overline{t}} = a^2 \frac{{y}_x - \hat{y}_{\overline{x}}}{h}
		\]
		\[
		\frac{y_{i}^{j+1}-2y_i^{j}+y_i^{j-1}}{\tau^2} = \frac{a}{h^2} ((y_{i+1}^{j} -y_{i}^{j} )-(y_i^{j+1} - y_{i-1}^{j+1}))
		\]
		
		
		9. Схема П (неявная)
		\[
		\frac{y_{i-1}^{j+1}-2y_{i-1}^{j}+y_{i-1}^{j-1}}{\tau^2} + \frac{y_{i+1}^{j+1}-2y_{i+1}^{j}+y_{i+1}^{j-1}}{\tau^2} = a^2\frac{y_{i-1}^{j+1} -2y_i^{j+1} +y_{i+1}^{j+1}}{h^2}
		\]
		
		
		10.Схема П (явная)
		\[
		\frac{y_{i-1}^{j+1}-2y_{i-1}^{j}+y_{i-1}^{j-1}}{\tau^2} + \frac{y_{i+1}^{j+1}-2y_{i+1}^{j}+y_{i+1}^{j-1}}{\tau^2} = a^2\frac{y_{i-1}^{j} -2y_i^{j} +y_{i+1}^{j}}{h^2}
		\]
		
		
		11. Схема перевернутая П(неявная)

		\[
		\frac{y_{i-1}^{j+1}-2y_{i-1}^{j}+y_{i-1}^{j-1}}{\tau^2} + \frac{y_{i+1}^{j+1}-2y_{i+1}^{j}+y_{i+1}^{j-1}}{\tau^2} = a^2\frac{y_{i-1}^{j-1} -2y_i^{j-1} +y_{i+1}^{j-1}}{h^2}
		\]
		\item Постройте разностную схему с весами для уравнения
		колебаний струны. Является ли такая схема устойчивой
		и монотонной?
		\newline 
		{\bfseries Ответ. } 
		\[
		y_{t\overline{t}} = a^2( \sigma \hat{y}_{x\overline{x}} + (1- \sigma)y_{x\overline{x}})
		\]
		Исследуем на монотонность
		
		
		Обозначим $\gamma = \frac{\tau^2a^2}{h^2}$
		\[
		\hat{y} - 2y+ \check{y} = \gamma(\sigma(\hat{y_{+1}} -2\hat{y} +\hat{y_{-1}}) + (1-\sigma)(y_{+1}-2y+y_{-1}))
		\]
		выразим относительно ведущего элемента $\hat{y}$ и получим, что коэффициент перед $\check{y}$ отрицательный. Не выполняется УПК. Схема безусловно не монотонна монотонна.
		
		
		
		Рассмотрим 9 точечную схему с весами
		\[
		y_{t\overline{t}} = a^2( \sigma \hat{y}_{x\overline{x}} + (1-2 \sigma)y_{x\overline{x}} + \sigma \check{y}_{x\overline{x}})
		\]
		
	
		$$\frac{\hat{y}-2y+\check{y}}{\tau^2}=a^2\left(\sigma \frac{\hat{y_{+1}}-2\hat{y}+\hat{y_{-1}}}{h^2}+(1-2\sigma)\frac{y_{+1}-2y+y_{-1}}{h^2}+\sigma \frac{\check{y_{+1}}-2\check{y}+\check{y_{-1}}}{h^2}\right)$$
		
		
		Воспользуемся необходимым спектральным признаком устойчивости. $y^j_i=\rho^j e^{i \tilde{i}\varphi}$



		\[
		\rho^2 - 2 \frac{1 - 2 \left( \frac{a\tau}{h} \right)^2 \sin^2 \frac{\varphi}{2} (1 - 2\sigma)}{1 + 4\sigma \left( \frac{a\tau}{h} \right)^2 \sin^2 \frac{\varphi}{2}} \rho + 1 = 0.
		\]
		
		Для устойчивости необходимо $\rho_1 \le 1, \, \rho_2 \le 1$. Однако если корни действительные, то в силу теоремы Виета $\rho_1 * \rho_2 = 1$, и следовательно один из корнец неизбежно будет больше 1. Поэтому необходимо(но не достаточно), чтобы корни были мнимыми. Т.е $D \leq 1$
		
		\[
		\left| \frac{1 - 2 \left( \frac{a\tau}{h} \right)^2 \sin^2 \frac{\varphi}{2} (1 - 2\sigma)}{1 + 4\sigma \left( \frac{a\tau}{h} \right)^2 \sin^2 \frac{\varphi}{2}} \right| \leq 1,
		\]
		
		Решая, получаем
		\[
		\sigma \geq \frac{1}{4} - \frac{h^2}{4a^2 \tau^2 \sin^2 \frac{\varphi}{2}}.
		\]
		

		\[
		\sigma \geq \frac{1}{4} - \frac{h^2}{4a^2 \tau^2}.
		\]
		
	
		Проверим схему на монотонность.
		
		\[
		\frac{\hat{y} - 2y + \check{y}}{\tau^2} = a^2 \left( \sigma \frac{\hat{y}_{+1} - 2\hat{y} + \hat{y}_{-1}}{h^2} + (1 - 2\sigma) \frac{y_{+1} - 2y + y_{-1}}{h^2} + \sigma \frac{\check{y}_{+1} - 2\check{y} + \check{y}_{-1}}{h^2} \right),
		\]
		
		
		Разрешая, относительно ведущего элемента $\hat{y}$
		
		
		
		\[
		\hat{y}(1+2\gamma \sigma) = \gamma \sigma(\hat{y}_{+1}\hat{y}_{-1}+\check{y}_{+1} + \check{y}_{-1}) + \gamma (1-2\sigma) (y_{-1} + y_{+1}) + y(2- 2\gamma(1-2\sigma)) + \check{y} (-1-2\gamma \sigma)
		\]
		
		Коэффициент перед $\check{y}$ отрицательный, значит схема не монотонна.
		\item Предложите способ контроля точности полученного решения.
		\newline 
		{\bfseries Ответ. } 
		Пусть у нас есть решение на сетке с шагом $h$ и $\tau$.
		Хотим узнать на сколько точным оно получилось. 
		Запишем погрешность на двух сетках: $(h, \tau)$ и $(h/2, \tau/2)$:
		\begin{gather*}
			\|z_{h, \tau}\| = C_1 \tau^2 + C_2 h^2 \\ 
			\|z_{h/2, \tau/2}\| = \frac{1}{4} (C_1\tau^2 + C_2 h^2) 
		\end{gather*}
		Далее запишем 
		\begin{equation*}
			\|z_{h, \tau}\| = \|y_{(h)}^{(\tau)} - u\| = \|y_{(h)}^{(\tau)} - y_{(h/2)}^{(\tau/2)} + 
			y_{(h/2)}^{(\tau/2)} - u\| \leqslant \| y_{(h)}^{(\tau)} - y_{(h/2)}^{(\tau/2)}\| + \|z_{h/2, \tau/2}\|,
		\end{equation*}
		где $y_{(h)}^{(\tau)}$ и $y_{(h/2)}^{(\tau/2)}$ -- решения на текущем временном слое, полученное 
		с помощью соответствующей сетки.
		Также вычитая погрешности на разных сетках, получим:
		\begin{equation*}
			\|z_{h, \tau}\| - \|z_{h/2, \tau/2}\| = \frac{3}{4} \|z_{h, \tau}\|
		\end{equation*}
		Тогда подставляя в неравенство
		\begin{equation*}
			\|z_{h, \tau}\| \leqslant \frac{4}{3} \| y_{(h)}^{(\tau)} - y_{(h/2)}^{(\tau/2)}\|
		\end{equation*}
		Получили оценку для погрешности на текущей сетке. Теперь рассмотрим способы получения 
		решения на сетке $(h/2, \tau/2)$:
		\begin{enumerate}
			\item Путем параллельного расчета на двух сетках.
			\item Получить промежуточные точки по $x$ можно с помощью сплайн-интерполяции. Получить 
			 промежуточные точки по $t$ можно с помощью выполнения двух шагов по времени с предыдущего временного слоя.
		\end{enumerate}
		\item Приведите пример трехслойной схемы для уравнения теплопроводности. Как реализовать вычисления по такой разностной схеме? Является ли эта схема устойчивой?
		\newline 
		{\bfseries Ответ. } 
		Схема Дюфорта-Франкела:
		$$\frac{\hat{y}-\check{y}}{2 \tau}=a^2 \frac{y_{+1}-\hat{y_i}-\check{y_i}+y_{-1}}{h^2}$$
		
		Временной слой при $t = 0$ получаем из начального условия. Далее, чтобы 
		вычислить слой $t = \tau$, разложим по формуле Тейлора:
		\begin{equation*}
			y_i^1 = y_i^0 + \tau y'_i + O(\tau^2),
		\end{equation*}
		где производная вычислена от $\varphi(x)$, где $\varphi(x) = y(x,0)$.
		
		Далее будем исследовать на устойчивость. Сделаем замену:
		\begin{equation*}
			y_i^j = \rho^j e^{i \tilde{i} \varphi},
		\end{equation*}
		где $\tilde{i} = \sqrt{-1}$.
		Подставим в схему с учетом $\gamma = \frac{\tau a^2}{h^2}$
		\begin{equation*}
			\rho + 1/\rho = 2 \gamma \left(e^{\tilde{i} \varphi } - \rho - 
			\rho^{-1} + e^{- \tilde{i} \varphi }\right) = 2 \gamma (2 \cos \varphi 
			- (\rho + 1/\rho))
		\end{equation*}
		В итоге получаем 
		\begin{equation*}
			\rho = \frac{2 \gamma \cos \varphi \pm + \sqrt{4 \gamma ^ 2 \cos^2\varphi - 4\gamma^2 +1}}
			{1+2\gamma}
		\end{equation*}
		Получаем следующую оценку для $|\rho|$:
		\begin{equation*}
			|\rho| \leqslant \frac{|2 \gamma \cos \varphi| + \sqrt{1-4\gamma^2 \sin^2 \varphi}}{1+2\gamma}
		\end{equation*}
		Рассмотрим 2 случая:
		
		Пусть $1-4\gamma^2 \sin^2\varphi > 0$:
		\begin{equation*}
			|\rho| < 1 - \frac{1}{1+2\gamma} +  \frac{\sqrt{1-4\gamma^2}}{1+2\gamma} < 1 - \frac{1}{1+2 \gamma} + \frac{1}{1+2\gamma} < 1
		\end{equation*}

		Пусть  $1-4\gamma^2 \sin^2\varphi < 0$:

		\begin{equation*}
			|\rho|^2 = \frac{4\gamma \cos^2 \varphi + (1 - 4\gamma^2 \sin^2 \varphi)}{(1+2\gamma)^2} = 
			\frac{1 + 4 \gamma (\cos^2 \varphi - \sin^2 \varphi)}{1+4\gamma^2 + 4\gamma } < \frac{1+4\gamma}{1+4\gamma^2 + 4\gamma} < 1
		\end{equation*}
		Тогда получаем, что схема безусловно устойчива.

	\end{enumerate}
	\section{Порядки}
	\begin{table}[H]
		\centering
		\caption{Порядок аппроксимации}
		\begin{tabular}{|c|c|}
			\hline
			Шаг сетки & $p$ \\ \hline
			$h, \tau$ & --- \\ \hline 
			$q h, q \tau$ &  2.810873236323846 \\ \hline 
			$q^2 h, q^2 \tau$ & 2.220212471275546 \\ \hline 
			$q^3 h, q^3 \tau$ & 1.99999999999998 \\ \hline 
			$q^4 h, q^{4} \tau$ & 2.00006100061483 \\ \hline 
	
		\end{tabular}
	\end{table}
	% \begin{table}[H]
	% 	\centering
	% 	\caption{Порядок аппроксимации по $\tau$}
	% 	\begin{tabular}{|c|c|}
	% 		\hline
	% 		Шаг сетки & $p$ \\ \hline
	% 		$h, \tau$ & --- \\ \hline 
	% 		$q h, q \tau$ &  2.810873236323846 \\ \hline 
	% 		$q^2 h, q^2 \tau$ & 1.999605244154222 \\ \hline 
	% 		$q^3 h, q^3 \tau$ & 1.989237035192481 \\ \hline 
	% 		$q^4 h, q^{4} \tau$ & 1.99925 \\ \hline 
	% 	\end{tabular}
	% \end{table}
	% \section{Дополнительное задание}

\end{document}

\documentclass{article}
\usepackage[T2A]{fontenc}
\usepackage{epigraph}
\usepackage[english, russian]{babel} % языковой пакет
\usepackage{amsmath,amsfonts,amssymb} %математика
\usepackage{mathtools}
\usepackage[oglav,spisok,boldsect,eqwhole,figwhole,hyperref,hyperprint,remarks,greekit]{../../style/fn2kursstyle}
\usepackage[utf8]{inputenc}
\usepackage[]{tkz-euclide}
\usepackage{algpseudocode}
\usepackage{pgfplots}
\usepackage{tikz-3dplot}
\usepackage[oglav,spisok,boldsect,eqwhole,figwhole,hyperref,hyperprint,remarks,greekit]{./style/fn2kursstyle}
\usepackage{multirow}
\usepackage{supertabular}
\usepackage{multicol}
\usepackage{tikz}
\usepackage{pgfplots}
\usepackage{float}
\usepackage{graphicx}
\pgfplotsset{compat=1.9}
\usepackage[svgnames]{pstricks}
\usepackage{pst-solides3d} 
\usepackage{multirow}
\usepackage{hhline}
\usepackage{slashbox}
\usepackage{pdflscape}
\usepackage{array} 
\graphicspath{{../../style/}{../}}

  



\newcommand{\cond}{\mathop{\mathrm{cond}}\nolimits}
\newcommand{\rank}{\mathop{\mathrm{rank}}\nolimits}
% Переопределение команды \vec, чтобы векторы печатались полужирным курсивом
\renewcommand{\vec}[1]{\text{\mathversion{bold}${#1}$}}%{\bi{#1}}
\newcommand\thh[1]{\text{\mathversion{bold}${#1}$}}
%Переопределение команды нумерации перечней: точки заменяются на скобки
\renewcommand{\labelenumi}{\theenumi)}
\newtheorem{theorem}{Теорема}
\newtheorem{define}{Определение}
\tdplotsetmaincoords{60}{115}
\pgfplotsset{compat=newest}

\title{Методы численного решения обыкновенных
дифференциальных уравнений}
\author{Н.\,О.~Акиньшин}
\group{ФН2-61Б}
\date{2025}
\supervisor{А.\,С.~Джагарян}



\begin{document}
    \maketitle
    \newpage
    \tableofcontents
    \newpage

    \section{Контрольные вопросы}
    \begin{enumerate}
        \item Сформулируйте условия существования и единственности
        решения задачи Коши для обыкновенных дифференциальных уравнений. Выполнены ли они для вашего варианта
        задания?
        \newline
        {\bfseries Ответ. } 
        \noindent Теорема (о существовании и единственности Задачи Коши)
	
	
	Рассмотрим задачу Коши
	\begin{equation}
		\begin{split}
			u'(t)&=f(t,u) \\
			u(t_0)&=u_0
		\end{split}
	\end{equation}
	
	Условие 1 гарантирует существование решения Задачи Коши, а условие 2 единственность. Таким образом существует и единственно решение Задачи Коши при выполнении следующих условий
	
	
	1) Функция $f(t,u)$ определена и непрерывна в прямоугольнике \[D=\{(t, u):\, |t-t_0| \le a; \,|u_i-u_{0,i}| \le b, \, i =\overline{1,n}\}\]
	в прямоугольнике все компоненты $|f_i| \le M$
	
	2)Функция $f(t,u)$ липшиц-непрерывна с постоянной $L$ по переменным $u_1,u_2,\ldots \, u_n$:
	\[
	|f(t, u^{(1)}) - f(t, u^{(2)})| \le L \sum_{i=1}^{n}|u_i^{(1)}-u_i^{(2)}|
	\]
	
	Тогда существует единственное решение задачи Коши на участке 
	
	\[
	|t-t_0|\le \min\{a,b/M,1/L\} 
	\]
	
	
	Замечание. Верно следующее включение $C^1 \subset C_L \subset C$
        \item Что такое фазовое пространство? Что называют фазовой
        траекторией? Что называют интегральной кривой?
        \newline
        {\bfseries Ответ. } 
        Пусть дана система 
        \begin{equation*}
            \dot{u_i} = f(t, \mathbf{u}), \quad i = 1,\ldots, n
        \end{equation*}
        Тогда решение системы  для любого момента 
        времени $t \in T\subset \mathbb{R}$ можно изобразить в $n-$мерном пространстве, которое 
        называется фазовым. 
        Интегральной кривой называют кривую $\Gamma = \{(t, \mathbf{u}) | \mathbf{u} =
         (u_1, u_2, \ldots, u_n), t \in T\}$ $\subset \mathbb{R}^{n+1}$, где $u_i$ -- решение системы.
        Фазовой траекторией называется кривая $\tilde{\Gamma} = \{(u_1, u_2, \ldots, u_n)\}$ $\subset \mathbb{R}^{n}$, где $u_i$ -- решение системы.
        
        \item Каким порядком аппроксимации и точности обладают
        методы, рассмотренные в лабораторной работе?
        \newline
        {\bfseries Ответ. } 
        Порядок аппроксимации:
	
	
	1)Метод Эйлера(явный, неявный) имеют 1ый порядок аппроксимации.
	
	
	2)Симметричная схема имеет 2ой порядок аппроксимации.
	
	
	3)Методы Рунге-Кутты, Адамса-Башфорта, прогноз-коррекция имеют 4ый порядок аппроксимации

	
	Порядок точности: 
	
	Явный метод Эйлера является частным случаем метода Рунге-Кутты, а неявный Эйлер, симметричная схема и метод Адамса-Башфорта является частным случаем линейного m-шагового разностного метода, следовательно требуется рассмотреть сходимость только методов Рунге-Кутты и линейного m-шагового разностного метода.
	
	
	Рассмотрим метод Рунге-Кутты
	
	
	\[
	\frac{y_{n+1}-y_n}{\tau} = \sum_{j=1}^{m} \sigma_j k_j,\, k_j = f \left(t_n+a_j \tau,\, y_n + \sum_{i=1}^{j-1}b_{ji}\tau k_i\right), \, j= \overline{1,m}; a_1=0
	\]
	Запишем приближенное решение в виде $y_n=z_n+u_n$, $u_n$ - точное решение, $z_n$ - погрешность. Тогда 
	\[
	\frac{z_{n+1}-z_n}{\tau} = \psi_h^{(1)} + \psi_h^{(2)} 
	\]
	где 
	\[
	\psi_h^{(1)} = \sum_{i=1}^{m}\sigma_i k_i(t_n, u_n,\tau) - \frac{u_{n+1}-u_n}{\tau};
	\]
	
	\[
	\psi_h^{(2)} = \sum_{i=1}^{m}\sigma_i \left ( k_i(t_n, y_n,\tau) -k_i(t_n, u_n,\tau)\right);
	\]
	
	
	Теорема о погрешности методов Рунге-Кутты.
	
	
	Пусть правая часть ОДУ удовлетворяет условию Липшица по второму аргументу с постоянной L. Тогда для погрешности метода Рунге-Кутты при $n \tau \le T $ справедлива оценка
	\[
	|z_n| = |y_n-u(t_n)| \le T e^{\alpha T} \max\limits_{0\le j \le n-1}|\psi_j^{(1)}|
	\]
	где $\alpha = \sigma L m(1+Lb\tau)^{m-1}$, где $\sigma = \max \limits_{1\le i \le m} |\sigma_i|; \, b = \max|b_{ij}|$
	
	
	Следствие. При выполнении условий теоремы порядок точности метода Рунге-Кутты совпадает с порядком аппроксимации. 
	
	
	Это следует из оценки погрешности и равномерной по $\tau$ ограниченности $\alpha$
	
	\[
	\alpha = \sigma L m(1+Lb\tau)^{m-1} \le \sigma L m e^{(m-1)Lb\tau} \le \sigma L m e^{(m-1)LbT}
	\]
	
	
	Рассмотрим линейный m-шаговый разностный метод решения ОДУ
	
	
	\[
	\frac{a_0y_n+a_1y_{n-1}+ \ldots a_my_{n-m}}{\tau} = b_0f_n+b_1f_{n-1}+ \ldots + b_mf_{n-m}
	\]
	
	Теорема о погрешности  m-шаговых разностных методов. Пусть разностный m-шаговый метод удовлетворяет условию корней и $|f'_y| \le L$. Тогда для любого $m\tau \le t_n =n\tau \le T$ при достаточно малом $\tau$ выполнена оценка
	\[
	|y_n-u(t_n)| \le M \left( \max\limits_{0\le j\le m-1}|y_j-u(t_j)|+\max \limits_{m\le j \le n}|\psi_{h,j}^{(1)}|\right)
	\]
	
	
	Таким образом методы имеют следующий порядок точности зависит от того с какой точностью будут найдены первые m значений. Пусть k порядок точности с которой найдены первые m значение т.е. 
	\[
	\max\limits_{0\le j\le m-1}|y_j-u(t_j)| = O(\tau^k)
	\]
	Тогда порядок точности равен $\min\{k, p\}$
        \item Какие задачи называются жесткими? Какие методы предпочтительны для их решения? Какие из рассмотренных
        методов можно использовать для решения жестких задач?
        \newline
        {\bfseries Ответ. } 
        
        \item Как найти $y_1$, $y_2$, $y_3$, чтобы реализовать алгоритм прогноза
        и коррекции (1.18)?
        \newline
        {\bfseries Ответ. } 
        1) Можно использовать любой из численных методов решения ОДУ требующих только начальное приближение. Например метод Рунге-Кутты или метод Эйлера и.т.д
	
	
	2)Можно раскладывать в ряд Тейлора в предыдущей точке т.е. 
	\[
	\vec{y_1} \approx \vec{u}(t_1) = \vec{u}(t_0) + \tau \vec{u'}(t_0)
	\]
	\[
	\vec{y_2} \approx \vec{u}(t_2) = \vec{u}(t_1) + \tau \vec{u'}(t_1)
	\]
	\[
	\vec{y_3} \approx \vec{u}(t_3) = \vec{u}(t_2) + \tau \vec{u'}(t_2)
	\]
	Производные можно найти из ОДУ $u'=f(t,u)$
	
	3)Можно раскладывать в ряд Тейлора в начальной точке т.е. 
	\[
	\vec{y_1} \approx \vec{u}(t_1) = \vec{u}(t_0) + \tau \vec{u'}(t_0)
	\]
	\[
	\vec{y_2} \approx \vec{u}(t_2) = \vec{u}(t_0) + 2\tau \vec{u'}(t_0)
	\]
	\[
	\vec{y_3} \approx \vec{u}(t_3) = \vec{u}(t_0) + 3\tau \vec{u'}(t_0)
	\]
	
	
	Замечание 2 и 3 по сути является методом Эйлера
        \item Какой из рассмотренных алгоритмов является менее трудоемким? Какой из рассмотренных алгоритмов позволяет достигнуть заданную точность, используя наибольший
        шаг интегрирования? Какие достоинства и недостатки рассмотренных алгоритмов вы можете указать?
        \newline
        {\bfseries Ответ. } 
        \item Какие алгоритмы, помимо правила Рунге, можно использовать для автоматического выбора шага?
        \newline
        {\bfseries Ответ. } 
        
    \end{enumerate}




\end{document}

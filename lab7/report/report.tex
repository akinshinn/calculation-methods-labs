\documentclass{article}
\usepackage[T2A]{fontenc}
\usepackage{epigraph}
\usepackage[english, russian]{babel} % языковой пакет
\usepackage{amsmath,amsfonts,amssymb} %математика
\usepackage{mathtools}
\usepackage[oglav,spisok,boldsect,eqwhole,figwhole,hyperref,hyperprint,remarks,greekit]{../../style/fn2kursstyle}
\usepackage[utf8]{inputenc}
\usepackage[]{tkz-euclide}
\usepackage{algpseudocode}
\usepackage{pgfplots}
\usepackage{tikz-3dplot}
\usepackage[oglav,spisok,boldsect,eqwhole,figwhole,hyperref,hyperprint,remarks,greekit]{./style/fn2kursstyle}
\usepackage{multirow}
\usepackage{supertabular}
\usepackage{multicol}
\usepackage{tikz}
\usepackage{pgfplots}
\usepackage{float}
\usepackage{graphicx}
\pgfplotsset{compat=1.9}
\usepackage[svgnames]{pstricks}
\usepackage{pst-solides3d} 
\usepackage{multirow}
\usepackage{hhline}
\usepackage{slashbox}
\usepackage{pdflscape}
\usepackage{array} 
\graphicspath{{../../style/}{../}}

  


\newcommand{\cond}{\mathop{\mathrm{cond}}\nolimits}
\newcommand{\rank}{\mathop{\mathrm{rank}}\nolimits}
% Переопределение команды \vec, чтобы векторы печатались полужирным курсивом
\renewcommand{\vec}[1]{\text{\mathversion{bold}${#1}$}}%{\bi{#1}}
\newcommand\thh[1]{\text{\mathversion{bold}${#1}$}}
%Переопределение команды нумерации перечней: точки заменяются на скобки
\renewcommand{\labelenumi}{\theenumi)}
\newtheorem{theorem}{Теорема}
\newtheorem{define}{Определение}
\tdplotsetmaincoords{60}{115}
\pgfplotsset{compat=newest}

\title{Численное решение краевых задач для одномерного уравнения теплопроводности}
\author{Н.\,О.~Акиньшин}
\group{ФН2-61Б}
\date{2025}
\supervisor{А.\,С.~Джагарян}



\begin{document}
    \maketitle
    \newpage
    \tableofcontents
    \newpage

    \section{Контрольные вопросы}
	\begin{enumerate}
		\item Дайте определения терминам: корректно поставленная задача, понятие аппроксимации дифференциальной задачи
		разностной схемой, порядок аппроксимации, однородная
		схема, консервативная схема, монотонная схема, устойчивая разностная схема (условно/абсолютно), сходимость.
		\newline
		{\bfseries Ответ. } 
		Пусть дана задача 
		\begin{align*}
			Au = f \  \text{ в } G, \quad  Ru = \mu \  \text{ на } \partial G,
		\end{align*}
		для которой известна разностная схема 
		\begin{align*}
			A_h y = \phi \text{ в } G_h, \quad R_h y = \nu \text{ на }\partial G_h.
		\end{align*}
		Разностная схема $A_h y = \varphi, \ R_h y = \nu$ называется корректной, если её решение существует,
		единственно и непрерывно зависит от входных данных.
		Погрешность аппроксимации данной разностной схема определяется как $\Psi_h = (\varphi - f_n) + 
		((Av)_h - A_h v_h)$.
		Погрешность аппроксимации граничных и начальных условий $\chi_h = (\nu - \mu_n) + 
		((R v)_h - R_h v_h) $.

		Разностная схема аппроксимирует исходную задачу, если 
		$\|\Psi_h\| \to 0, \|\chi_h\| \to 0 \text{ при } {h \to 0}$.
		
		Аппроксимация имеет порядок $p$, если$\|\Psi_h \| = O(h^p),\  \|\chi_h\| = O(h^p), 
		\text{ при } h \to 0$.
		Аппроксимацию называют условной, если она имеет место только при наличии некоторой зависимости 
		между шагами по разным направлениям и безусловной в противном случае. 

		Разностная схема называется устойчивой, если её решение непрерывно зависит от входных данных. 
		Устойчивость называется условной, если её наличие зависит от соотношения шагов сетки по разным 
		направлениям, и безусловной в противном случае.

		Схема называется консервативной, если её решение удовлетворяет дискретному аналогу закона сохранения,
		 присущего данной задаче.

		 Разностная схема называется монотонной, если она удовлетворяет аналогу принципа максимума, присущего
		 исходной задаче.
		\item Какие из рассмотренных схем являются абсолютно устойчивыми? Какая из рассмотренных схем позволяет вести
		расчеты с более крупным шагом по времени?
		\newline 
		{\bfseries Ответ. } 
		Рассмотрим устойчивость схемы с весами. 
		\begin{equation*}
			y_t - k \tau \sigma y_{t\bar{x}x} = k y_{\bar{x}x}
		\end{equation*}
		Разложим $y$ по собственным функциям $\mu_j$, которые соответствуют собственному 
		значению $\lambda_j$.
		Отметим, что на $\lambda_j$ накладываются следующие условия 
		\begin{equation*}
			\frac{9}{l^2} \leqslant \lambda_j \leqslant \frac{4}{h^2}
		\end{equation*}
		Тогда $y$ представимо 
		\begin{equation*}
			y = \sum_{j=1}^{n-1} c_j \mu_j
		\end{equation*}
		Подставим в схему 
		\begin{equation*}
			\sum_{j=1}^{n-1} \left(c_{j,t} + k \tau \sigma \lambda_j c_{j,t} + 
			k \lambda_j c_j \right)\mu_j = 0
		\end{equation*}
		Тогда для $j=1,\ldots, n-1$:
		\begin{equation*}
			(1+k\tau\sigma \lambda_j ) c_{j,t} + k \lambda_j c_j = 0
		\end{equation*}
		Далее получаем 
		\begin{equation*}
			(1 + k \tau \sigma \lambda_j ) \hat{c_j} = (1+k \tau \sigma) c_j - 
			k \tau \lambda_j c_j 
		\end{equation*}
		Тогда 
		\begin{equation*}
			\hat{c_j} = \rho_j c_j,
		\end{equation*}
		где 
		\begin{equation*}
			\rho_j = \frac{1+k \tau \sigma \lambda_j - k \tau \lambda_j}{1+k \tau 
			\sigma \lambda_j}	
		\end{equation*}
		при $|\rho_j| \leqslant 1$ справедливо
		\begin{equation*}
			\|\hat{y}\|^2 = \sum_{j=1}^{n-1} \hat{c_j}^2 = \sum_{j=1}^{n-1} \rho_j^2 c_j^2
			\leqslant \|y\|^2
		\end{equation*}
		Заметим, что 
		\begin{equation*}
			1 + k \tau \sigma \lambda_j > 0
		\end{equation*}
		Тогда 
		\begin{equation*}
			\sigma > \frac{-1}{k \tau \lambda_j}
		\end{equation*}
		Запишем ограничения на $\rho_j$
		\begin{equation*}
			k \tau \lambda_j \leqslant 2 + 2 k \tau \sigma \lambda_j
		\end{equation*}		
		Тогда 
		\begin{equation*}
			\sigma \geqslant \frac{1}{2} - \frac{h^2}{4k\tau}
		\end{equation*}
		Из рассмотренных схем смешанная и неявная схемы абсолютно устойчивы, явная схема ---
		условно устойчива. 
		В силу того, что смешанная схема имеет порядок аппроксимации $O(t^2 + h^2)$, то она 
		позволяет совершать больший шаг по времени, чем другие схемы.
		\item Будет ли смешанная схема (2.15) иметь второй порядок
		аппроксимации при $a_i = \dfrac{2 K(x_i) K(x_{i-1})}{K(x_i) + K(x_{i-1})}$
		\newline
		{\bfseries Ответ. } 
		Достаточное условие второго порядка аппроксимации:
		\begin{equation*}
			a_i = K(x_i) - \frac{h}{2}K'(x_i) + O(h^2)
		\end{equation*}
		Тогда 
		\begin{equation*}
			\frac{2K_i(K_i - h K'_i + \frac{h^2}{2}K''_i + O(h^3))}{2K_i - h K'_i + \frac{h^2}{2} K''_i + O(h^3)}=
			K_i - \frac{h}{2}K'_i + O(h^2)
		\end{equation*}
		Далее, домножая на знаменатель, получаем 
		\begin{equation*}
			2 K^2_i - 2 h K_i K'_i  + O(h^2) = 2 K^2_i - h K_i K'_i - h K_i K'_i + O(h^2)
		\end{equation*}
		Из чего следует второй порядок аппроксимации.
		\item Какие методы (способы) построения разностной аппроксимации граничных условий 
		(2.5), (2.6) с порядком точности $O(\tau + h^2), \ O(\tau^2 + h^2), \ O(\tau^2 + h)$
		\newline
		{\bfseries Ответ. } 
		Пусть заданы следующие граничные условия 
		\begin{gather*}
			- K(0) u_x (0, t) = P_1(t) \\ 
			-K(L) u_x (L, t) = P_2(t) 
		\end{gather*}
		Схема для внутренних точек имеет второй порядок по каждой переменной, значит
		будем рассматривать только границы. Поскольку границы имеет один род 
		граничных условий, то ограничимся рассмотрением левого граничного условия. 

		Пользуясь интегро-интерполяционным методом, получим следующие выражение 
		\begin{equation*}
			\int \limits_0^{x_{1/2}} (u(x, t_{j+1}) - u(x, t_j)) dx 
			=  \int \limits_{t_j}^{t_{j+1}}
			(K_{1/2} u_x (x_{1/2}, t) + P_1(t)) dt
		\end{equation*}
		Тогда аппроксимируем интегралы с помощью левых прямоугольников. 
		Получаем 
		\begin{equation*}
			\frac{1}{2\tau}(\hat{y}_{1}-  y_0) = K_{1/2} \frac{y_{1} - y_0}{h^2} 
			+ \frac{1}{h^2} P_1
		\end{equation*}
		Вычисление погрешности аппроксимации приводит к 
		\begin{equation*}
			\psi_h = O(\tau + h^2)
		\end{equation*}
		Если аппроксимировать интегралы с помощью схемы с весами и принять вес 
		$\sigma = 1/2$, то получим 
		\begin{equation*}
			\hat{y_0} (-\frac{1}{2}\tau K_{1/2} - h^2/2) + 
			\hat{y_1} (\frac{1}{2} \tau K_{1/2}) = 
			- \tau (1/2) (K_{1/2} (y_1 - y_0) + h P_1) - \frac{h^2}{2} y_0 + 
			\tau \frac{1}{2} h \hat{P_1}
		\end{equation*}
		Разложим в ряд в $y_0$ и получим следующий порядок аппроксимации 
		\begin{equation*}
			\psi_h = O(\tau^2 + h^2)
		\end{equation*}

		Далее будем считать, что $K = const$ и $P_1 = const, \ P_2 = const$, тогда 
		\begin{gather*}
			u_x (0, t) = Q_1 \\ 
			u_x (L, t) = Q_2 
		\end{gather*}
		Если в выражении, полученным интегро-интерполяционным методом 
		интеграл по $x$ аппроксимировать через левые прямоугольники, а 
		интеграл по $t$ аппроксимировать через центральные, то получим 
		\begin{equation*}
			\frac{1}{2\tau}(\hat{y}_{1}-  y_0) = \frac{1}{2}\tau (K \frac{\hat{y}_1
			- \hat{y}_0}{h^2}) -  \frac{1}{2}\tau (K \frac{y_1
			- y_0}{h^2})
		\end{equation*}
		Тогда получим 
		\begin{equation*}
			\psi_h = O(\tau^2 + h)
		\end{equation*}
		\item При каких $h, \ \tau, \ \sigma$ смешанная схема монотонна? Проиллюстрируйте результатами расчетов свойства монотонных
		и немонотонных разностных схем.
		\newline
		{\bfseries Ответ. } 
		Запишем расчетную схему в каноническом виде 
		\begin{equation*}
			\hat{y} (\frac{c \rho }{\tau} + \frac{\sigma a_{i+1}}{h^2} + \frac{a_i \sigma}{h^2}) 
			= y \left(\frac{c \rho}{\tau}- \frac{1-\sigma}{h^2} - \frac{a_i}{h^2}(1-\sigma)\right) 
			+ y_{+1} \left(\frac{1-\sigma}{h^2}a_{i+1}\right) + \hat{y}_{+1} \frac{\sigma}{h^2} a_{i+1}
			+ \hat{y}_{-1} \frac{a_i \sigma}{h^2}.
		\end{equation*}
		Заметим, что $F \equiv 0$, следовательно $Ly = 0$.
		Далее получаем следующую систему 
		\begin{equation*}
			\begin{cases}
				\frac{c \rho }{\tau} + \frac{\sigma a_{i+1}}{h^2} + \frac{a_i \sigma}{h^2} > 0, \\ 
				\frac{c \rho}{\tau}- \frac{1-\sigma}{h^2} - \frac{a_i}{h^2}(1-\sigma) > 0, \\
				\frac{1-\sigma}{h^2} a_i > 0, \\ 
				\frac{\sigma}{h^2} a_{i+1} > 0, \\ 
				\frac{a_i \sigma}{h^2} > 0.
			\end{cases}
		\end{equation*}
		В силу естественных условий на $a_i, \sigma, \tau, \rho, c, h$ получаем следующее условие 
		\begin{equation*}
			\frac{c \rho}{\tau}- \frac{1-\sigma}{h^2} - \frac{a_i}{h^2}(1-\sigma) > 0
		\end{equation*}
		Рассмотрим $D = A(x) - \sum\limits_{\xi \in S'(x)} B(\xi, \, x)$
		\begin{equation*}
			D = \frac{c \rho }{\tau} + \frac{\sigma a_{i+1}}{h^2} + \frac{a_i \sigma}{h^2} - 
			(\frac{c \rho}{\tau}- \frac{1-\sigma}{h^2} - \frac{a_i}{h^2}(1-\sigma)) -
			\frac{1-\sigma}{h^2} a_i - \frac{\sigma}{h^2} a_{i+1} - \frac{a_i \sigma}{h^2} \equiv 0
		\end{equation*}
		В итоге если выполнено условие $\frac{c \rho}{\tau}- \frac{1-\sigma}{h^2} - \frac{a_i}{h^2}(1-\sigma) > 0$, то по
		теореме о выполнении принципа максимума для расчетной схемы, данная схема будет монотонна. 
		\item Какие ограничения на $h$, $\tau$ и $\sigma$ накладывают условия
		устойчивости прогонки?
		\newline
		{\bfseries Ответ. } 
		Запишем расчетную схему в трехдиагональном виде 
		\begin{gather*}
			\hat{y}_{-1} (\tau \sigma k_{-1/2}) + \hat{y} (-\tau \sigma k_{+1/2}
			 - \tau \sigma k_{-1/2} - c \rho h^2) + \hat{y}_{i+1} (\tau \sigma k_{+1/2})= \\=
			 -c \rho h^2 y - \tau (1-\sigma) (k_{+1/2}(y_{+1/2 - y}) - k_{-1/2} (y - y_{-1/2}))
		\end{gather*}
		Из теоремы об устойчивости и корректности прогонки 
		\begin{equation*}
			|b_i| \geqslant |a_i| + |c_i|
		\end{equation*}
		Тогда получаем
		\begin{equation*}
			\tau \sigma k_{+1/2}
			+ \tau \sigma k_{-1/2} + c \rho h^2 \geqslant \tau \sigma k_{-1/2} + \tau \sigma k_{+1/2}
		\end{equation*}
		Что верно всегда. 
		Теперь рассмотрим положительность $|b_1| > 0$. Запишем расчетную схему для левого граничного условия
		\begin{equation*}
			\hat{y} (- \tau \sigma k_{1/2} + \tau \sigma \frac{b}{\beta_1} - \frac{c \rho h^2}{2}) + 
			\hat{y}_1 \tau \sigma k_{1/2} = -\tau (1-\sigma) (k_{1/2}(y_1 - y) + \frac{h}{\beta_1} (\alpha_1 y - P_1)) 
			- \frac{c \rho h^2}{2} y + \tau \sigma \frac{h}{\beta_1} P_1
		\end{equation*}
		Заметим, что всегда выполнено условие $|b_1| > 0$. 
		Условие неравенства 0 для $c_i$ очевидно. Заметим, что $|b_1| > |c_1|$.
		Также накладывается следующее условие
		\begin{equation*}
			|- \tau \sigma k_{1/2} + \tau \sigma \frac{b}{\beta_1} - \frac{c \rho h^2}{2}| > |\tau \sigma k_{1/2}|
		\end{equation*}
		\begin{equation*}
			\tau \sigma k_{1/2} +  \frac{c \rho h^2}{2} -\tau \sigma \frac{b}{\beta_1} >  \tau \sigma k_{1/2}
		\end{equation*}
		\begin{equation*}
			\frac{c \rho h^2}{2} -\tau \sigma \frac{b}{\beta_1} >0
		\end{equation*}
		Что эквивалентно 
		Теперь проверим для правого граничного условия условие $|b_n| > |a_n|$
		Запишем расчетную схему для правого граничного условия
		\begin{gather*}
			\hat{y}_{n-1} \tau \sigma k_{n-1/2} +\\+  \hat{y} (-\tau \sigma k_{n-1/2} - \tau \sigma \frac{h}{\beta_2} \alpha_2 - \frac{c \rho h^2}{2}) = 
			f_n
		\end{gather*}
		Из схемы видно, что 
		\begin{equation*}
			|b_n| > |a_n|
		\end{equation*}
		Тогда по теореме о корректности и устойчивости метода прогонки 
		\item В случае $K = K(u)$ чему равно количество внутренних итераций, если итерационный процесс вести до сходимости,
		а не обрывать после нескольких первых итераций?
		\newline
		{\bfseries Ответ. } 
		\begin{table}[h]
			\centering
			\caption{Среднее количество итераций}
			\begin{tabular}{|c|c|c|}
				\hline
				$\varepsilon = 1e-6$ &$\varepsilon = 1e-8$& $\varepsilon = 1e-10$ \\
				\hline
				3& 4 & 5\\
				\hline
			\end{tabular}
		\end{table}
		\item Для случая $K = K(u)$ предложите способы организации
		внутреннего итерационного процесса или алгоритмы, заменяющие его.
		\newline
		{\bfseries Ответ. } 
		Вместо использования метода простой итерации, можно пользоваться любым итерационным методом 
		для решения нелинейных систем. Например, методом Ньютона.
		Рассмотрим достоинства и недостатки некоторых итерационных методов.
		Среди достоинств метода Ньютона можно отметить быструю сходимость (сойдется за 1-2 итерации), однако
		среди недостатков стоит отметить долгую сходимость в случае попадания начального условия на плато функции. 
		Среди достоинств метода простой итерации отметим простоту реализации, явное вычисление следующих итераций, однако 
		к недостаткам относится более медленная сходимость по сравнению с методом Ньютона. 

	\end{enumerate}

\section{Дополнительные вопросы}
\begin{enumerate}
	\item Записать общее решение уравнения теплопроводности.
	\newline
	{\bfseries Ответ. } 
	Пусть дана следующая задача
	\begin{equation*}
		\begin{cases}
			u_t = a^2 u_{xx} + f(x, t), \\ 
			u(x, 0) = \phi(x), \\
			\alpha_1 u(0, t) - \beta_1 u_t(0, t) = \mu_1(t), \\ 
			\alpha_2 u(l, t) + \beta_2 u_t(l, t) = \mu_2(t), \\ 
			x  \in [0, l], t >0
		\end{cases}
	\end{equation*}
	Тогда её решение представимо в следующем виде 
	\begin{equation*}
		u(x, t) =  \sum_{n = 0}^{\infty} \varphi_n e^{-a^2 \lambda_n t}X_n(t) + 
		\int_{0}^{t} \int_{0}^{l} f(\xi, \tau) G(x, \xi, t - \tau) d\xi d\tau,
	\end{equation*}
	\begin{equation*}
		G(x, \xi, t - \tau) = \sum_{n=0}^{\infty} e^{-a^2 \lambda_n (t - \tau)} X_n(\xi) X_n(x)
	\end{equation*}
	\item Записать прогочные коэффициенты когда слева задан поток, а справа -- постоянная температура
	\newline 
	{\bfseries Ответ. } 
	Запишем аппроксимированное левое граничное условие
	\begin{equation*}
		(-\frac{1}{2} c \rho h^2 - \tau  \sigma K_{1/2}) \hat{y_0} + \tau \sigma K_{1/2}  \hat{y}_1 = 
		-\tau (1-\sigma ) ( K_{1/2} (y_1 - y_0) - h \check{p}_1 / \beta_1) - \frac{1}{2}  c  \rho  h^2 \check{y}_0 + \tau  \sigma  h  \check{p}_2/ \beta_1
	\end{equation*}
	Представляя
	\begin{equation*}
		\hat{y}_0 = \varkappa \hat{y}_1 + \mu,
 	\end{equation*}
	получим 
	\begin{gather*}
		\varkappa = - \frac{\tau \sigma K_{1/2}}{-\frac{1}{2} c \rho h^2 - \tau  \sigma K_{1/2}}, \\
		\mu = \frac{-\tau (1-\sigma ) ( K_{1/2} (y_1 - y_0) - h \check{p}_1 / \beta_1) - \frac{1}{2}  c  \rho  h^2 \check{y}_0 + \tau  \sigma  h  \check{p}_2/ \beta_1}{(-\frac{1}{2} c \rho h^2 - \tau  \sigma K_{1/2})}
	\end{gather*}
	Для правого граничного условия:
	\begin{equation*}
		\hat{y}_{n} = p_2
	\end{equation*}
	В силу того, что на правом конце ничего аппроксимировать не нужно, то получим
	\begin{gather*}
		\varkappa = 0, \\
		\mu = p_2
	\end{gather*}
	\item Пример неконсервативной расчетной схемы 
	\newline 
	{\bfseries Ответ. } Рассмотрим следующую задачу 
	\begin{gather*}
		(k(x)u_x)_x = 0, \quad 0<x<1, \\
		u(0) = 1, \\ 
		u(1) = 0
	\end{gather*}
	Пусть 
	\[
	k(x) = \begin{cases}
		2, \, 0 \leqslant x \le 1/2; \\ 
		1, \, 1/2 \leqslant x \le 1.
	\end{cases}
	\]
	Ее точное решение имеет вид
	\[
	u(x) = 
		\begin{cases}
			1-\frac{2x}{3}, \, 0 \le x < 0.5; \\ 
			\frac{4(1-x)}{3}, \, 0.5 \le x < 1.
		\end{cases}
	\]
	В точке разрыва коэффициента справедливы соотношения 
	$u |_{0.5-0} = u |_{0.5+0},
	\, (ku_x)|_{0.5-0} = (ku_x)|_{0.5+0}$
	Для решения исходной задачи всюду, кроме точки $x = 0.5$, 
	справедливо уравнение $ku_{xx} =0$. 
	Для такого уравнение можно записать разностную схему $ky_{\overline{x}x}$.
	Если узел сетки не поподает на точку разрыва коэффициента, то такая
	процедура выглядит на первый взгляд вполне приемлимой.
	Коэффициент $k=1$ или $k=2$ в зависимости от точки.
	Уравнение $ky_{\overline{x}x}=0$ можно на него разделить
	и в результате получить совершенно точное решение такой
	разностной схемы $y=1-x, \, y_i = 1-ih,\, i=0,1, \ldots, n, \, nh = 1$.
	При этом норма разности решений точной и приближенной задач равна
	\[
	||y-u_h||_C = 2/3 - 1/2 = 1/6 \not\rightarrow 0 
	\]
	при $h \rightarrow 0$
\end{enumerate}

\section{Порядки}
\begin{table}[H]
	\centering
	\caption{Порядок аппроксимации для схемы $\sigma = 1$}
	\begin{tabular}{|c|c|}
		\hline
		Шаг сетки & $p$ \\ \hline
		$h, \tau$ & --- \\ \hline 
		$q h, q^2 \tau$ &  3.18332 \\ \hline 
		$q^2 h, q^4 \tau$ & 2.13314 \\ \hline 
		$q^3 h, q^8 \tau$ & 2.03735 \\ \hline 
		$q^4 h, q^{16} \tau$ & 2.00962 \\ \hline 

	\end{tabular}
\end{table}

\begin{table}[H]
	\centering
	\caption{Порядок аппроксимации для схемы $\sigma = 0.5$}
	\begin{tabular}{|c|c|}
		\hline
		Шаг сетки & $p$ \\ \hline
		$h, \tau$ & --- \\ \hline 
		$q h, q \tau$ &  5.33143 \\ \hline 
		$q^2 h, q^2 \tau$ & 1.98742 \\ \hline 
		$q^3 h, q^3 \tau$ & 1.99696 \\ \hline 
		$q^4 h, q^{4} \tau$ & 1.99925 \\ \hline 

	\end{tabular}
\end{table}

\begin{table}[H]
	\centering
	\caption{Порядок аппроксимации для схемы $\sigma = 0$}
	\begin{tabular}{|c|c|}
		\hline
		Шаг сетки & $p$ \\ \hline
		$h, \tau$ & --- \\ \hline 
		$q h, q^2 \tau$ &  2.01596 \\ \hline 
		$q^2 h, q^4 \tau$ & 2.00398 \\ \hline 
		$q^3 h, q^8 \tau$ & 2.00099 \\ \hline 
		$q^4 h, q^{16} \tau$ & 2.00025 \\ \hline 

	\end{tabular}
\end{table}

\end{document}

\documentclass{article}
\usepackage[T2A]{fontenc}
\usepackage{epigraph}
\usepackage[english, russian]{babel} % языковой пакет
\usepackage{amsmath,amsfonts,amssymb} %математика
\usepackage{mathtools}
\usepackage[oglav,spisok,boldsect,eqwhole,figwhole,hyperref,hyperprint,remarks,greekit]{../../style/fn2kursstyle}
\usepackage[utf8]{inputenc}
\usepackage[]{tkz-euclide}
\usepackage{algpseudocode}
\usepackage{pgfplots}
\usepackage{tikz-3dplot}
\usepackage[oglav,spisok,boldsect,eqwhole,figwhole,hyperref,hyperprint,remarks,greekit]{./style/fn2kursstyle}
\usepackage{multirow}
\usepackage{supertabular}
\usepackage{multicol}
\usepackage{tikz}
\usepackage{pgfplots}
\usepackage{float}
\usepackage{graphicx}
\pgfplotsset{compat=1.9}
\usepackage[svgnames]{pstricks}
\usepackage{pst-solides3d} 
\usepackage{multirow}
\usepackage{hhline}
\usepackage{slashbox}
\usepackage{pdflscape}
\usepackage{array} 
\graphicspath{{../../style/}{../}}

  


\newcommand{\cond}{\mathop{\mathrm{cond}}\nolimits}
\newcommand{\rank}{\mathop{\mathrm{rank}}\nolimits}
% Переопределение команды \vec, чтобы векторы печатались полужирным курсивом
\renewcommand{\vec}[1]{\text{\mathversion{bold}${#1}$}}%{\bi{#1}}
\newcommand\thh[1]{\text{\mathversion{bold}${#1}$}}
%Переопределение команды нумерации перечней: точки заменяются на скобки
\renewcommand{\labelenumi}{\theenumi)}
\newtheorem{theorem}{Теорема}
\newtheorem{define}{Определение}
\tdplotsetmaincoords{60}{115}
\pgfplotsset{compat=newest}

\title{Численное решение краевых задач для одномерного уравнения теплопроводности}
\author{Н.\,О.~Акиньшин}
\group{ФН2-61Б}
\date{2025}
\supervisor{А.\,С.~Джагарян}



\begin{document}
    \maketitle
    \newpage
    \tableofcontents
    \newpage

    \section{Контрольные вопросы}
	\begin{enumerate}
		\item Дайте определения терминам: корректно поставленная задача, понятие аппроксимации дифференциальной задачи
		разностной схемой, порядок аппроксимации, однородная
		схема, консервативная схема, монотонная схема, устойчивая разностная схема (условно/абсолютно), сходимость.
		\newline
		{\bfseries Ответ. } 
		Пусть дана задача 
		\begin{align*}
			Au = f \  \text{ в } G, \quad  Ru = \mu \  \text{ на } \partial G,
		\end{align*}
		для которой известна разностная схема 
		\begin{align*}
			A_h y = \phi \text{ в } G_h, \quad R_h y = \nu \text{ на }\partial G_h.
		\end{align*}
		Разностная схема $A_h y = \varphi, \ R_h y = \nu$ называется корректной, если её решение существует,
		единственно и непрерывно зависит от входных данных.
		Погрешность аппроксимации данной разностной схема определяется как $\Psi_h = (\varphi - f_n) + 
		((Av)_h - A_h v_h)$.
		Погрешность аппроксимации граничных и начальных условий $\chi_h = (\nu - \mu_n) + 
		((R v)_h - R_h v_h) $.

		Разностная схема аппроксимирует исходную задачу, если 
		$\|\Psi_h\| \to 0, \|\chi_h\| \to 0 \text{ при } {h \to 0}$.
		
		Аппроксимация имеет порядок $p$, если$\|\Psi_h \| = O(h^p),\  \|\chi_h\| = O(h^p), 
		\text{ при } h \to 0$.
		Аппроксимацию называют условной, если она имеет место только при наличии некоторой зависимости 
		между шагами по разным направлениям и безусловной в противном случае. 

		Разностная схема называется устойчивой, если её решение непрерывно зависит от входных данных. 
		Устойчивость называется условной, если её наличие зависит от соотношения шагов сетки по разным 
		направлениям, и безусловной в противном случае.

		Схема называется консервативной, если её решение удовлетворяет дискретному аналогу закона сохранения,
		 присущего данной задаче.

		 Разностная схема называется монотонной, если она удовлетворяет аналогу принципа максимума, присущего
		 исходной задаче.
		\item Какие из рассмотренных схем являются абсолютно устойчивыми? Какая из рассмотренных схем позволяет вести
		расчеты с более крупным шагом по времени?
		\newline 
		{\bfseries Ответ. } 
		\item Будет ли смешанная схема (2.15) иметь второй порядок
		аппроксимации при $a_i = \dfrac{2 K(x_i) K(x_{i-1})}{K(x_i) + K(x_{i-1})}$
		\newline
		{\bfseries Ответ. } 
		Достаточное условие второго порядка аппроксимации:
		\begin{equation*}
			a_i = K(x_i) - \frac{h}{2}K'(x_i) + O(h^2)
		\end{equation*}
		Тогда 
		\begin{equation*}
			\frac{2K_i(K_i - h K'_i + \frac{h^2}{2}K''_i + O(h^3))}{2K_i - h K'_i + \frac{h^2}{2} K''_i + O(h^3)}=
			K_i - \frac{h}{2}K'_i + O(h^2)
		\end{equation*}
		Далее, домножая на знаменатель, получаем 
		\begin{equation*}
			2 K^2_i - 2 h K_i K'_i  + O(h^2) = 2 K^2_i - h K_i K'_i - h K_i K'_i + O(h^2)
		\end{equation*}
		Из чего следует второй порядок аппроксимации.
		\item Какие методы (способы) построения разностной аппроксимации граничных условий 
		(2.5), (2.6) с порядком точности $O(\tau + h^2), \ O(\tau^2 + h^2), \ O(\tau^2 + h)$
		\newline
		{\bfseries Ответ. } 
		\item При каких $h, \ \tau, \ \sigma$ смешанная схема монотонна? Проиллюстрируйте результатами расчетов свойства монотонных
		и немонотонных разностных схем.
		\newline
		{\bfseries Ответ. } 
		Запишем расчетную схему в каноническом виде 
		\begin{equation*}
			\hat{y} (\frac{c \rho }{\tau} + \frac{\sigma a_{i+1}}{h^2} + \frac{a_i \sigma}{h^2}) 
			= y \left(\frac{c \rho}{\tau}- \frac{1-\sigma}{h^2} - \frac{a_i}{h^2}(1-\sigma)\right) 
			+ y_{+1} \left(\frac{1-\sigma}{h^2}a_{i+1}\right) + \hat{y}_{+1} \frac{\sigma}{h^2} a_{i+1}
			+ \hat{y}_{-1} \frac{a_i \sigma}{h^2}.
		\end{equation*}
		Заметим, что $F \equiv 0$, следовательно $Ly = 0$.
		Далее получаем следующую систему 
		\begin{equation*}
			\begin{cases}
				\frac{c \rho }{\tau} + \frac{\sigma a_{i+1}}{h^2} + \frac{a_i \sigma}{h^2} > 0, \\ 
				\frac{c \rho}{\tau}- \frac{1-\sigma}{h^2} - \frac{a_i}{h^2}(1-\sigma) > 0, \\
				\frac{1-\sigma}{h^2} a_i > 0, \\ 
				\frac{\sigma}{h^2} a_{i+1} > 0, \\ 
				\frac{a_i \sigma}{h^2} > 0.
			\end{cases}
		\end{equation*}
		В силу естественных условий на $a_i, \sigma, \tau, \rho, c, h$ получаем следующее условие 
		\begin{equation*}
			\frac{c \rho}{\tau}- \frac{1-\sigma}{h^2} - \frac{a_i}{h^2}(1-\sigma) > 0
		\end{equation*}
		Рассмотрим $D = A(x) - \sum\limits_{\xi \in S'(x)} B(\xi, \, x)$
		\begin{equation*}
			D = \frac{c \rho }{\tau} + \frac{\sigma a_{i+1}}{h^2} + \frac{a_i \sigma}{h^2} - 
			(\frac{c \rho}{\tau}- \frac{1-\sigma}{h^2} - \frac{a_i}{h^2}(1-\sigma)) -
			\frac{1-\sigma}{h^2} a_i - \frac{\sigma}{h^2} a_{i+1} - \frac{a_i \sigma}{h^2} \equiv 0
		\end{equation*}
		В итоге если выполнено условие $\frac{c \rho}{\tau}- \frac{1-\sigma}{h^2} - \frac{a_i}{h^2}(1-\sigma) > 0$, то по
		теореме о выполнении принципа максимума для расчетной схемы, данная схема будет монотонна. 
		\item Какие ограничения на $h$, $\tau$ и $\sigma$ накладывают условия
		устойчивости прогонки?
		\newline
		{\bfseries Ответ. } 
		Запишем расчетную схему в трехдиагональном виде 
		\begin{gather*}
			\hat{y}_{-1} (\tau \sigma k_{-1/2}) + \hat{y} (-\tau \sigma k_{+1/2}
			 - \tau \sigma k_{-1/2} - c \rho h^2) + \hat{y}_{i+1} (\tau \sigma k_{+1/2})= \\=
			 -c \rho h^2 y - \tau (1-\sigma) (k_{+1/2}(y_{+1/2 - y}) - k_{-1/2} (y - y_{-1/2}))
		\end{gather*}
		Из теоремы об устойчивости и корректности прогонки 
		\begin{equation*}
			|b_i| \geqslant |a_i| + |c_i|
		\end{equation*}
		Тогда получаем
		\begin{equation*}
			\tau \sigma k_{+1/2}
			+ \tau \sigma k_{-1/2} + c \rho h^2 \geqslant \tau \sigma k_{-1/2} + \tau \sigma k_{+1/2}
		\end{equation*}
		Что верно всегда. 
		Теперь рассмотрим положительность $|b_1| > 0$. Запишем расчетную схему для левого граничного условия
		\begin{equation*}
			\hat{y} (- \tau \sigma k_{1/2} + \tau \sigma \frac{b}{\beta_1} - \frac{c \rho h^2}{2}) + 
			\hat{y}_1 \tau \sigma k_{1/2} = -\tau (1-\sigma) (k_{1/2}(y_1 - y) + \frac{h}{\beta_1} (\alpha_1 y - P_1)) 
			- \frac{c \rho h^2}{2} y + \tau \sigma \frac{h}{\beta_1} P_1
		\end{equation*}
		Заметим, что всегда выполнено условие $|b_1| > 0$. 
		Условие неравенства 0 для $c_i$ очевидно. Заметим, что $|b_1| > |c_1|$.
		Также накладывается следующее условие
		\begin{equation*}
			|- \tau \sigma k_{1/2} + \tau \sigma \frac{b}{\beta_1} - \frac{c \rho h^2}{2}| > |\tau \sigma k_{1/2}|
		\end{equation*}
		\begin{equation*}
			\tau \sigma k_{1/2} +  \frac{c \rho h^2}{2} -\tau \sigma \frac{b}{\beta_1} >  \tau \sigma k_{1/2}
		\end{equation*}
		\begin{equation*}
			\frac{c \rho h^2}{2} -\tau \sigma \frac{b}{\beta_1} >0
		\end{equation*}
		Что эквивалентно 
		Теперь проверим для правого граничного условия условие $|b_n| > |a_n|$
		Запишем расчетную схему для правого граничного условия
		\begin{gather*}
			\hat{y}_{n-1} \tau \sigma k_{n-1/2} +\\+  \hat{y} (-\tau \sigma k_{n-1/2} - \tau \sigma \frac{h}{\beta_2} \alpha_2 - \frac{c \rho h^2}{2}) = 
			f_n
		\end{gather*}
		Из схемы видно, что 
		\begin{equation*}
			|b_n| > |a_n|
		\end{equation*}
		Тогда по теореме о корректности и устойчивости метода прогонки 
		\item В случае $K = K(u)$ чему равно количество внутренних итераций, если итерационный процесс вести до сходимости,
		а не обрывать после нескольких первых итераций?
		\newline
		{\bfseries Ответ. } 
		\item Для случая $K = K(u)$ предложите способы организации
		внутреннего итерационного процесса или алгоритмы, заменяющие его.
		\newline
		{\bfseries Ответ. } 
		Вместо использования метода простой итерации, можно пользоваться любым итерационный методом 
		для решения нелинейных систем. Например, метод Ньютона.
	\end{enumerate}



\end{document}

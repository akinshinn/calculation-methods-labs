\documentclass{article}
\usepackage[T2A]{fontenc}
\usepackage{epigraph}
\usepackage[english, russian]{babel} % языковой пакет
\usepackage{amsmath,amsfonts,amssymb} %математика
\usepackage{mathtools}
\usepackage[oglav,spisok,boldsect,eqwhole,figwhole,hyperref,hyperprint,remarks,greekit]{../../style/fn2kursstyle}
\graphicspath{{../../style/}}



\newcommand{\cond}{\mathop{\mathrm{cond}}\nolimits}


\title{Прямые методы решения систем
линейных алгебраических уравнений}
\author{Н.\,О.~Акиньшин}
\group{ФН2-51Б}
\date{2024}
\supervisor{А.\,С.~Джагарян}



\begin{document}
    \maketitle
    \newpage
    \tableofcontents
    \newpage

    \section{Исходные данные}
    \begin{enumerate}
        \item Тестовый пример 1
        \begin{equation}
            \begin{dcases}
                x_1 + x_2 + x_3 + x_4 = 4, \\
                    x_2 + x_3 + x_4 = 3, \\ 
                        x_3 + x_4 = 2, \\
                            x_4 = 1.
            \end{dcases}
        \end{equation}
        Точное решение: $x = (1, 1, 1, 1)^T$. \newline
        $\cond_1 A = 8, \  \cond_\infty A = 8$
        \item Тестовый пример 2
            \begin{equation}
                \begin{dcases}
                    x_4 = 1, \\
                    x_3 + x_4 = 2, \\
                    x_2 + x_3 + x_4 = 3, \\ 
                    x_1 + x_2 + x_3 + x_4 = 4. \\
                \end{dcases}
            \end{equation}
            Точное решение: $x = (1, 1, 1, 1)^T$. \newline
            $\cond_1 A = 8, \  \cond_\infty A = 8$
        \item Тестовый пример 3
            \begin{equation}
                \begin{dcases}
                    x_1 + x_2 + x_3 + x_4 = 4, \\ 
                    2x_1 + 3x_2 + 3x_3 + 3x_4 = 11, \\ 
                    2 x_1 + 4x_2 + 4x_3 + 4x_4 = 15, \\ 
                    4 x_1 + 5 x_2 + 6 x_3 + 7 x_4 = 22
                \end{dcases}
            \end{equation}
        Матрица не совместна, $\cond A = \infty$.
        \item Тестовый пример 4
        \begin{equation}
            \begin{dcases}
                10x_1 + 6x_2 + 2x_3 = 25, \\
                5x_1  + x_2 - 2x_3 + 4x_4 = 14, \\ 
                3x_1 + 5x_2 + x_3 - x_4 = 10, \\
                6x_2 -2x_3 +2x_4 = 8
            \end{dcases}
        \end{equation}
        Точное решение: $x = (2, 1, -0.5, 0.5)^T$. \newline
            $\cond_1 A \approx 240.55, \  \cond_\infty A \approx 269.19$
        \item Тестовый пример 5
        \begin{equation}
            \begin{dcases}
                28.589x_1 -0.008x_2+ 2.406x_3+ 19.24x_4 = 30.459, \\ 
                14.436x_1 -0.001x_2 + 1.203x_3 + 9.624x_4 = 18.248, \\
                120.204x_1 -0.032x_2 + 10.024x_3 +  80.144x_4 = 128.156, \\
                -57.714x_1 + 0.016x_2 -4.812x_3 -38.478x_4 = -60.908.
            \end{dcases}
        \end{equation}
        Точное решение: $x = (1, 1000, -20, 3)^T$. \newline
            $\cond_1 A \approx 122414849.9, \  \cond_\infty A \approx 109686235.3.$
        \item Система варианта 1
        \begin{equation}
            \begin{dcases}
                16.3820x_1 -2.0490x_2 -41.8290x_3 + 16.3920x_4 = 33.6130\\
                307.6480x_1 -38.4660x_2 -840.3660x_3 + 312.5280x_4 = 710.3420 \\
                0.4560x_1 -0.0570x_2 -1.1770x_3 + 0.4560x_4 = 0.9490 \\ 
                23.2720x_1 -2.9090x_2 -66.3090x_3 + 23.8720 = 57.6730
            \end{dcases}
        \end{equation}
        \item Система варианта 2
        \begin{equation}
            \begin{dcases}
                31.2000x_1 -1.3200x_2 -7.6800 x_3 + 4.0900 x_4 = -83.3200 \\
                7.2300 x_1 -126.0000 x_2 + 7.1400 x_3 + 3.0400 x_4 = 38.9000 \\ 
                9.4900 x_1 + 6.4000 x_2 + 6.0000 x_3 + 8.4500 x_4 = -56.7000 \\ 
                2.6800 x_1 -3.2900 x_2 + 0.2800 x_3 + 13.4000 x_4 = -504.0900
            \end{dcases}
        \end{equation}  
    \end{enumerate}

\section{Краткое описание используемых алгоритмов}
\subsection{Метод Гаусса}
Метод Гаусса --- это алгоритм решения систем линейных уравнений. 
Основная идея заключается в преобразовании системы уравнений к треугольному виду с 
помощью элементарных преобразований строк матрицы, а затем нахождении решения методом обратного 
хода.
\subsection{QR-разложение}
QR-разложение --- это метод разложения матрицы на произведение двух матриц: ортогональной матрицы 
Q и верхнетреугольной матрицы R.
\section{Контрольные вопросы}
\begin{enumerate}
    \item Каковы условия применимости метода Гаусса без выбора
    и с выбором ведущего элемента?
    \newline
    {\bfseries Ответ}.
    В случае, если применяется метод Гаусса без выбора ведущего элемента, то условие применимости 
    алгоритма заключается в $a_{ii} \neq 0$, где $a_{ii}$ -- элемент матрицы коэффициентов $A$.
    Если же используется выбор ведущего элемента, то достаточно требовать от матрицы наличия
    в соответствующих строке или столбце наличие ненулевых элементов.
    \item Докажите, что если $\det A \neq 0$, то при выборе главного
    элемента в столбце среди элементов, лежащих не выше
    главной диагонали, всегда найдется хотя бы один элемент,
    отличный от нуля.
    \newline
    {\bfseries Доказательство}. Пусть $\det A \neq 0$ и все элементы, лежащие не выше главной диагонали, равны 0,
    то есть
    \begin{equation*}
        A = 
        \begin{pmatrix}
            0 &  a_{12} & \dots & a_{1n} \\ 
            0 & 0 & \dots & a_{2n} \\ 
            \vdots &&& \\
            0 & \hdotsfor{2} & 0
        \end{pmatrix}
    \end{equation*}    
    Тогда матрица $A$ имеет верхнетреугольный вид, тогда $\det A = a_{11} \cdot a_{22} \cdot \ldots \cdot a_{nn}$.
    Получаем, что $\det A = 0$, что противоречит с изначальным условием, значит существует хотя бы 1 не нулевой элемент,
    который находиться не выше главной диагонали.
    Пусть $A_1 \sim A_2$ и $\det A \neq 0$, тогда $\det A_2 \neq 0$.
    Тогда пусть существует k ненулевых элементов в разных столбцах, находящихся ниже главной диагонали,
    причем $k < n$.
    Тогда переставим строки так, чтобы эти элементы находились на главной диагонали, причем 
    вычтем переставленные строки так, чтобы итоговая матрица имела нижнетреугольный вид. Заметим, что
    хотя бы 1 элемент на главной диагонале остался равным 0, следовательно $\det A = 0$ --
    противоречие, следовательно $k = n$.
    \item В методе Гаусса с полным выбором ведущего элемента
    приходится не только переставлять уравнения, но и менять нумерацию неизвестных. 
    Предложите алгоритм, позволяющий восстановить первоначальный порядок неизвестных.
    \newline
    {\bfseries Ответ}.
    Будем хранить перестановку переменных $x_i \rightarrow x_j$ в виде $(i, j)$.
    Тогда создадим переменную $permutations$, которая будет хранить все перестановки.

    В общем виде:
    \begin{equation*}
        permutations = ((n_1, n_2), (n_3, n_4), \ldots, (n_{k-1}, n_k)),
    \end{equation*}
    где $n_i$ - порядковый номер переменной.
    При добавлении новой нумерации двух переменных следует записать их в конец $permutations$.
    Для восстановления порядка, необходимо проделать замены в обратном порядке, записанном в $permutations$:
    \begin{equation*}
        recover = ((n_k, n_{k-1}), (n_{k-2}, n_{k-3}), \ldots, (n_2, n_1)).
    \end{equation*}
    Выполняя замены, записанные в $recover$, можно получить изначальный порядок переменных.
    \item Оцените количество арифметических операций, требуемых для QR-разложения произвольной матрицы $A$ размера $n\times n $.
    \newline
    {\bfseries Ответ}.
    \item Что такое число обусловленности и что оно характеризует?
    Имеется ли связь между обусловленностью и величиной
    определителя матрицы? Как влияет выбор нормы матрицы
    на оценку числа обусловленности?
    \newline
    {\bfseries Ответ}.
    Величину
    \begin{equation*}
        \cond A = \|A\| \|A^{-1}\|
    \end{equation*}
    называют числом обусловленности матрицы A. Оно показывает влияние погрешностей
    правой части $b$ на точное решение $x$. 
    
    Покажем связь между $\cond A$ и $\det A$. Пусть $\det A \neq 0$.
    \begin{equation*}
        \begin{dcases}
            A(x+\Delta x) = b + \Delta b, \\
            Ax = b
        \end{dcases}
    \end{equation*}
    Тогда справедливо
    \begin{equation*}
        A\Delta x = \Delta b
    \end{equation*}
    В силу неотрицательности определителя
    \begin{equation*}
        \Delta x = A^{-1}\Delta b
    \end{equation*}
    Тогда перейдем к неравенству
    \begin{equation*}
        \|\Delta x\| \leqslant \|A^{-1}\| \|\Delta b\|
    \end{equation*}
    \item Как упрощается оценка числа обусловленности, если мат-
    рица является:
    
    а) диагональной;
    б) симметричной;
    в) ортогональной;
    г) положительно определенной;
    д) треугольной?
    \newline
    {\bfseries Ответ}.
    a) Справедлива следующая оценка
    \begin{equation*}
        \cond A \geqslant \frac{| \lambda_{max}|}{|\lambda_{min}|},
    \end{equation*}
    но все собсвтенные числа находяться на диагонале в силу диагонального вида матрицы.
    Тогда
    \begin{equation*}
        \cond A \geqslant \frac{|max(a_{ii})|}{|min(a_{ii})|}, \quad i = 1,2,3 \ldots n
    \end{equation*}
    \item Применимо ли понятие числа обусловленности к вырожденным матрицам?
    \newline
    {\bfseries Ответ}. Нет, потому что $\cond A = \|A\| \|A^{-1}\|$, но если
    $\det A = 0$, то не существует $A^{-1}$.

    \item В каких случаях целесообразно использовать метод Гаусса,
    а в каких — методы, основанные на факторизации матрицы?

    \item Как можно объединить в одну процедуру прямой и обратный ход метода Гаусса? В чем достоинства и недостатки
    такого подхода?

    \item Объясните, почему, говоря о векторах, норму $\|\cdot\|_1$ часто
    называют октаэдрической, норму $\|\cdot\|_2$ — шаровой, а норму
    $\|\cdot\|_\infty$ — кубической.
\end{enumerate}
\end{document}
